\chapter{Resumen}

\vspace{0.5cm}

En astronomía, la edad es una de las propiedades estelares más difíciles de medir y la girocronología es una de las técnicas más prometedoras para esta tarea. Consiste en datar estrellas usando su periodo de rotación y relaciones lineales empíricas con otras propiedades estelares observadas. Sin embargo, estas aproximaciones no permiten reproducir todos los datos observados, lo que resulta en posibles desviaciones en la estimación de la edad.

En este contexto, se propone explorar el problema de la datación estelar utilizando la girocronología desde la perspectiva de la Inteligencia Artificial (IA). Para hacerlo, se presenta un estudio exhaustivo de evaluación comparativa de los modelos de IA en regresión de última generación entrenados y probados para la datación estelar mediante girocronología. También se explora en este trabajo cómo aplicar modelos de meta-learning para el problema de la datación estelar con pocos datos. Los resultados de ambos estudios son bastante prometedores, y abren una nueva línea de investigación en el campo de la estimación de la edad de las estrellas mediante IA.

\vspace{1cm}

\textbf{Palabras clave}: inteligencia artificial, aprendizaje automático, datación estelar, girocronología, astronomía.


%Introduce hoja en blanco
\newpage
\thispagestyle{empty}
\hspace*{0.5cm}
\newpage

\chapter{Abstract}

\vspace{0.5cm}
%TODO_DONE: he añadido un par de frases al de castellano, que debes traducir.

In astronomy, age is one of the most difficult stellar properties to measure, and gyrochronology is one of the most promising techniques for this task. It consists of dating stars using their period of rotation and empirical linear relationships with other observed stellar properties. However, these approximations do not allow to reproduce all the observed data, which results in possible deviations in the age estimate.

In this context, it is proposed to explore the problem of stellar dating using gyrochronology from the perspective of AI. To do so, we present a comprehensive benchmarking study of state-of-the-art regression Artificial Intelligence models trained and tested for gyrochronology stellar dating. How to apply meta-learning models to the stellar dating problem with little data is also explored in this paper. The results of both studies are quite promising, and open a new line of research in the field of estimating the age of stars using AI.

\vspace{1cm}

\textbf{Keywords}: artificial intelligence, machine learning, stellar dating, gyrochronology, astronomy.


%Introduce hoja en blanco
\newpage
\thispagestyle{empty}
\hspace*{0.5cm}
\newpage
