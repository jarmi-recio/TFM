\chapter{Resumen Extendido}
Debe incluir un resumen del trabajo de un máximo de 5 páginas. Resaltar aspectos fundamentales del desarrollo, los resultados más relevantes y las conclusiones.


En astronomía, la edad es una de las propiedades estelares más difíciles de medir y la girocronología es una de las técnicas más prometedoras para esta tarea. Consiste en datar estrellas usando su periodo de rotación y relaciones lineales empíricas con otras propiedades estelares observadas, como temperatura efectiva, paralelaje y/o colores fotométricos en diferentes bandas de paso, por ejemplo. Sin embargo, estas aproximaciones no permiten reproducir todos los datos observados, lo que resulta en posibles desviaciones en la estimación de la edad.

En este contexto, se propone explorar el problema de la datación estelar utilizando la girocronología desde la perspectiva de la IA. Técnicamente, se reemplazan otras combinaciones lineales y técnicas tradicionales con una aproximación de regresión de aprendizaje automático. Para hacerlo, se presenta un estudio exhaustivo de evaluación comparativa de los modelos de Inteligencia Artificial en regresón de última generación entrenados y probados para la datación estelar mediante girocronología.

Los experimentos revelan resultados prometedores, donde algunos modelos reportan un error promedio de $<0.5$ Gyr, lo que puede considerarse como un avance sobresaliente en el campo.

